\documentclass[landscape, a4paper]{article}
\usepackage[top=4mm,bottom=4mm,left=4mm,right=4mm]{geometry}
\usepackage[utf8]{inputenc}
\usepackage[T1]{fontenc}
\usepackage{multicol}
\usepackage{amsmath}
\usepackage{amssymb}
\usepackage{pdfpages}
\usepackage[T1]{fontenc}
\usepackage{enumitem}
\usepackage[framemethod=tikz]{mdframed}
\usepackage{sectsty}
\usepackage{empheq}
\usepackage{fancyvrb}

\newcommand{\zerodisplayskips}{%
  \setlength{\abovedisplayskip}{0pt}%
  \setlength{\belowdisplayskip}{0pt}%
  \setlength{\abovedisplayshortskip}{0pt}%
  \setlength{\belowdisplayshortskip}{0pt}
}
\newcommand{\Z}{\mathbb{Z}}
\newcommand{\Q}{\mathbb{Q}}
\newcommand{\R}{\mathbb{R}}
\newcommand{\C}{\mathscr{C}}
\newcommand{\Or}{\vee}
\renewcommand{\And}{\wedge}
\newcommand{\customsection}[1]{
    \vspace{-5pt}
    \section*{#1}
    \vspace{-4pt}
}
\newcommand{\boxedeq}[1]{
    \begin{empheq}[box={\fboxsep=6pt\fbox}]{align*}
        #1
    \end{empheq}
}

\appto{\normalsize}{\zerodisplayskips}
\appto{\small}{\zerodisplayskips}
\appto{\footnotesize}{\zerodisplayskips}
\setlength\parindent{0pt}
\sectionfont{\fontsize{12}{15}\selectfont}
\setlist[enumerate]{itemsep=0pt, topsep=3pt, parsep=0pt}


\begin{document}
\small
\begin{multicols*}{4}
  \boxed{\text{Done by Marcus Pang in 2021}}
  \customsection{Definitions}
  \begin{enumerate}
    \item computational process: set of activities in a computer,
          designed to achieve a desired result.
    \item primitive expressions: numerals, booleans, pre-declared functions
    \item program: expressions with semicolons
    \item pair: tuple of size 2 (array of size 2)
    \item list: null or pair whose tail is a list (nested arrays)
    \item tree of type $T$: null or pair whose head is of $T$ and whose tail is a list of type $T$
    \item binary tree: null or entry, left binary tree branch, and right binary tree branch
    \item stream of type $T$: null or pair whose head is of type $T$ and whose tail is a nullary function that returns a stream of type $T$
    \item array: stores a sequence of data elements
  \end{enumerate}
  \customsection{Order of Growth}
  Consider a growth function $f$ representing some program, then
  \begin{enumerate}
    \item $f(n)$ is big-theta of $g(n)$
          \begin{align*}
             & \iff f(n)\in \Theta(g(n))                                           \\
             & \iff f(n)\in \mathcal{O}(g(n)) \And f(n)\in \Omega(g(n))            \\
             & \iff \lim_{n\to\infty}\dfrac{f(n)}{g(n)}=c \text{ where }0<c<\infty
          \end{align*}
    \item $f(n)$ is big-O of $g(n)$\\
          $\iff$ for sufficiently large $n,$
          \[\exists c\in\R_{\geq 0}(f(n)\leq c\cdot g(n))\]
    \item $f(n)$ is big-$\Omega$ of $g(n)$\\
          $\iff$ for sufficiently large $n,$
          \[\exists c\in\R_{\geq 0}(f(n)\geq c\cdot g(n))\]
  \end{enumerate}

  \customsection{Substitution Model}
  Applicative-order reduction ("eager evaluation"):
  \begin{enumerate}
    \item primitive expressions: take the value
    \item operator combinations: evaluate operands, apply operator
    \item constant declaration: evaluate the value expression and replace the name by value
    \item function application: evaluate component expressions
          \begin{enumerate}
            \item primitive function $\implies$ apply the primitive function
            \item compound function $\implies$ substitute argument values for parameters in body of declaration
          \end{enumerate}
  \end{enumerate}
  Normal order reduction ("lazy evaluation"): arguments are not evaluated, they are passed to functions as expressions, and the expressions are only evaluated when needed\\
  Model breaks down when assignment statements are present as the model considers a constant/variable as just a name for the value and does not expect it to change.
  \customsection{Environment Model}
  Definitions:
  \begin{enumerate}
    \item environment: sequence of frames
    \item frame: contains bindings of values to names, points to enclosing environment (next one in sequence unless it is the global frame)
    \item extending environment: adding a new frame in existing frame
    \item global environment: single frame with bindings of primitive and pre-declared functions and constants
    \item program environment: directly extends global environment which contains all user programs
  \end{enumerate}
  Evaluating a name:
  \begin{enumerate}
    \item check value in current frame
    \item if not found, look in enclosing environment
    \item if not found, then name is unbound
  \end{enumerate}
  Evaluating a block:
  \begin{enumerate}
    \item if no constant and variable declaration, no frame is created
    \item else extend environment (create frame)
    \item store names of constants/variables with no values initially
    \item evaluate declarations and bind values to name
  \end{enumerate}
  Evaluating constant/variable declaration:
  \begin{enumerate}
    \item change binding of name to value in current frame
  \end{enumerate}
  Evaluating assignment statements:
  \begin{enumerate}
    \item search for name similar to evaluating names
    \item if name is constant/unbound return error
  \end{enumerate}
  Evaluating function applications:
  \begin{enumerate}
    \item evaluate arguments first in current environment
    \item extends environment in which function was created (create frame)
    \item store parameters in this new frame
    \item evaluate function body block
  \end{enumerate}
  Evaluating lambda expressions/function declarations
  \begin{enumerate}
    \item create function object: 1 circle points to body, 1 circle points to environment where it was created
  \end{enumerate}
  Use ":=" for constants and ":" for variables
  \customsection{Recursive Processes and CPS}
  All recursive processes can be converted to iterative processes
  \begin{enumerate}
    \item if recursive operations are always the same, apply the operation before the calling the recurisve function
    \item if order of operations do not matter, use helper function
    \item if order of operations matter, adjust to reverse order
    \item if that's too difficult, use CPS
  \end{enumerate}
  Continuation Passing Style: passing deferred operation as a function in extra argument to convert any recursive function

  \customsection{Interpreters and Compilers}
  Interpreter is a program that executes another program
  \boxedeq{\text{target}\\\text{source}}
  Compiler is a program that translates one language to another
  \boxedeq{\text{target}_1&\implies\text{target}_2\\&\text{source}}
  \customsection{Sorting}
  \scriptsize\begin{verbatim}
function bubblesort(xs) {
  const len = length(xs);
  for (let i = len - 1; i >= 1; i = i - 1) {
    let p = xs;
    for (let j = 0; j < i; j = j + 1) {
      if (head(p) > head(tail(p))) {
        const temp = head(p);
        set_head(p, head(tail(p)));
        set_head(tail(p), temp);
      }
      p = tail(p);
    }
  }
}
function selection_sort(xs) {
  if (is_null(xs)) {
    return xs;
  } else {
    const x = smallest(xs);
    return pair(
        x, 
        selection_sort(remove(x, xs))
    );
  }
}
function smallest(xs) {
    return is_null(tail(xs))
        ? head(xs)
        : head(xs) > smallest(tail(xs))
            ? smallest(tail(xs))
            : head(xs);
}
function merge_sort(xs) {
  if (is_null(xs) || is_null(tail(xs))) {
    return xs;
  } else {
    const mid = math_floor(length(xs) / 2   );
    return merge(
        merge_sort(take(xs, mid)), 
        merge_sort(drop(xs, mid))
    );
  }
}
function merge(xs, ys) {
  if (is_null(xs)) {
    return ys;
  } else if (is_null(ys)) {
    return xs;
  } else {
    const x = head(xs);
    const y = head(ys);
    return x < y 
        ? pair(x, merge(tail(xs), ys)) 
        : pair(y, merge(xs, tail(ys)));
  }
}
function take(xs, n) {
  return n === 0 
      ? null 
      : pair(head(xs), take(tail(xs), n - 1));
}
function drop(xs, n) {
  return n === 0 
      ? xs 
      : drop(tail(xs), n - 1);
}
function quicksort(xs) {
  return is_null(xs)
    ? null
    : is_null(tail(xs))
    ? xs
    : accumulate(
        append,
        null,
        list(
          quicksort(head(
            partition(tail(xs), head(xs)))
          ),
          list(head(xs)),
          quicksort(tail(
            partition(tail(xs), head(xs)))
          )
        )
      );
}
function partition(xs, p) {
  return pair(
    filter((a) => a <= p, xs),
    filter((a) => a > p, xs)
  );
}
function bubblesort_array(A) {
  const len = array_length(A);
  for (let i = len - 1; i >= 1; i = i - 1) {
    for (let j = 0; j < i; j = j + 1) {
      if (A[j] > A[j + 1]) {
        const temp = A[j];
        A[j] = A[j + 1];
        A[j + 1] = temp;
      }
    }
  }
}
function selection_sort_array(A) {
  const len = array_length(A);
  for (let i = 0; i < len - 1; i = i + 1) {
    let min_pos = find_min_pos(A, i, len - 1);
    swap(A, i, min_pos);
  }
}
function find_min_pos(A, low, high) {
  let min_pos = low;
  for (let j = low + 1; j <= high; j = j + 1) {
    if (A[j] < A[min_pos]) {
      min_pos = j;
    }
  }
  return min_pos;
}
function insertion_sort(A) {
  const len = array_length(A);
  for (let i = 1; i < len; i = i + 1) {
    let j = i - 1;
    while (j >= 0 && A[j] > A[j + 1]) {
      swap(A, j, j + 1);
      j = j - 1;
    }
  }
}
function merge_sort_array(A) {
  merge_sort_helper(A, 0, array_length(A) - 1);
}
function merge_sort_helper(A, low, high) {
  if (low < high) {
    const mid = math_floor((low + high) / 2);
    merge_sort_helper(A, low, mid);
    merge_sort_helper(A, mid + 1, high);
    merge(A, low, mid, high);
  }
}
function merge(A, low, mid, high) {
  const B = []; // temporary array
  let left = low;
  let right = mid + 1;
  let Bidx = 0;
  while (left <= mid && right <= high) {
    if (A[left] <= A[right]) {
      B[Bidx] = A[left];
      left = left + 1;
    } else {
      B[Bidx] = A[right];
      right = right + 1;
    }
    Bidx = Bidx + 1;
  }
  while (left <= mid) {
    B[Bidx] = A[left];
    Bidx = Bidx + 1;
    left = left + 1;
  }
  while (right <= high) {
    B[Bidx] = A[right];
    Bidx = Bidx + 1;
    right = right + 1;
  }
  for (let k = 0; k < high - low + 1; k = k + 1) {
    A[low + k] = B[k];
  }
}
\end{verbatim}
  \customsection{Searching}
  \scriptsize\begin{verbatim}
function binary_search_array(A, v) {
  let low = 0;
  let high = array_length(A) - 1;
  while (low <= high) {
    const mid = math_floor((low + high) / 2);
    if (v === A[mid]) {
      break;
    } else if (v < A[mid]) {
      high = mid - 1;
    } else {
      low = mid + 1;
    }
  }
  return low <= high;
}
\end{verbatim}
  \customsection{Memoization}
  \scriptsize\begin{verbatim}
function memoize(f) {
  const mem = [];
  function mf(x) {
    if (mem[x] !== undefined) {
      return mem[x];
    } else {
      const result = f(x);
      mem[x] = result;
      return result;
    }
  }
  return mf;
}
function mchoose(n, k) {
  if (read(n, k) !== undefined) {
    return read(n, k);
  } else {
    const result =
      k > n
        ? 0
        : k === 0 || k === n
        ? 1
        : mchoose(n - 1, k) + mchoose(n - 1, k - 1);
    write(n, k, result);
    return result;
  }
}
\end{verbatim}
  \customsection{Miscellaneous}
  \scriptsize\begin{verbatim}
function get_sublist(start, end, L) {
  function helper(pos, ys, result) {
    if (pos < start) {
      return helper(pos + 1, tail(ys), result);
    } else if (pos <= end) {
      return helper(pos + 1, tail(ys), 
                pair(head(ys), result));
    } else {
      return reverse(result);
    }
  }
  return helper(0, L, null);
}
function insertions(x, ys) {
  return map(
    (k) => append(take(ys, k), pair(x, drop(ys, k))),
    enum_list(0, length(ys))
  );
}
function permutations(ys) {
  return is_null(ys)
    ? list(null)
    : accumulate(
        append,
        null,
        map((x) => map((p) => pair(x, p), 
            permutations(remove(x, ys))), ys)
      );
}
function are_permutation(xs1, xs2) {
  return is_null(xs1) && is_null(xs2)
    ? true
    : !is_null(xs1) && !is_null(xs2)
    ? !is_null(member(head(xs1), xs2)) &&
      are_permutation(
        tail(xs1), 
        remove(head(xs1), xs2)
      )
    : false;
}
function combinations(xs, k) {
  if (k === 0) {
    return list(null);
  } else if (is_null(xs)) {
    return null;
  } else {
    const s1 = combinations(tail(xs), k - 1);
    const s2 = combinations(tail(xs), k);
    const x = head(xs);
    const has_x = map((s) => pair(x, s), s1);
    return append(has_x, s2);
  }
}
function remove_duplicates(lst) {
  return accumulate(
    (x, xs) => (is_null(member(x, xs)) 
            ? pair(x, xs) 
            : xs
    ),
    null,
    lst
  );
}
function subsets(xs) {
  return accumulate(
    (x, ss) =>
      append(
        ss,
        map((s) => pair(x, s), ss)
      ),
    list(null),
    xs
  );
}
function is_subset(S, T) {
  if (is_null(S)) {
    return true;
  } else if (is_null(T)) {
    return false;
  } else if (head(S) < head(T)) {
    return false;
  } else if (head(S) === head(T)) {
    return is_subset(tail(S), tail(T));
  } else {
    return is_subset(S, tail(T));
  }
}
function sort_ascending(A) {
  const len = array_length(A);
  for (let i = 1; i < len; i = i + 1) {
    const x = A[i];
    let j = i - 1;
    while (j >= 0 && A[j] > x) {
      A[j + 1] = A[j];
      j = j - 1;
    }
    A[j + 1] = x;
  }
}
function list_to_array(L) {
  const A = [];
  let i = 0;
  for (let p = L; !is_null(p); p = tail(p)) {
    A[i] = head(p);
    i = i + 1;
  }
  return A;
}
function array_to_list(A) {
  const len = array_length(A);
  let L = null;
  for (let i = len - 1; i >= 0; i = i - 1) {
    L = pair(A[i], L);
  }
  return L;
}
function count_pairs(x) {
  let pairs = null;
  function check(y) {
    if (!is_pair(y)) {
      return undefined;
    } else if (!is_null(member(y, pairs))) {
      return undefined;
    } else {
      pairs = pair(y, pairs);
      check(head(y));
      check(tail(y));
    }
  }
  check(x);
  return length(pairs);
}
\end{verbatim}
\end{multicols*}
\end{document}