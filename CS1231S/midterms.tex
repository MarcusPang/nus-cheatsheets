\documentclass[landscape, a4paper]{article}
% \usepackage[utf8]{inputenc}
% \usepackage[T1]{fontenc}
\usepackage{multicol}
% \usepackage{wrapfig}
\usepackage[top=2mm,bottom=2mm,left=2mm,right=2mm]{geometry}
\usepackage[framemethod=tikz]{mdframed}
\usepackage{pdfpages}
\usepackage{amsmath}
\usepackage{amsthm}
\usepackage{amssymb}
\usepackage{amsfonts}
\usepackage{mathrsfs}
\usepackage{tikz-cd}
\usepackage{enumitem}
\usepackage{centernot}

\newcommand{\Z}{\mathbb{Z}}
\newcommand{\Q}{\mathbb{Q}}
\newcommand{\R}{\mathbb{R}}
\newcommand{\C}{\mathscr{C}}
\newcommand{\Or}{\vee}
\newcommand{\true}{\textbf{true}}
\newcommand{\false}{\textbf{false}}
\renewcommand{\P}{\mathcal{P}}
\renewcommand{\and}{\wedge}
\newcommand{\zerodisplayskips}{%
  \setlength{\abovedisplayskip}{0pt}%
  \setlength{\belowdisplayskip}{0pt}%
  \setlength{\abovedisplayshortskip}{0pt}%
  \setlength{\belowdisplayshortskip}{0pt}
}
\newcommand{\customsection}[1]{%
    \vspace*{-16pt}%
    \section*{#1}%
    \vspace*{-5pt}%
}

\appto{\normalsize}{\zerodisplayskips}
\appto{\small}{\zerodisplayskips}
\appto{\footnotesize}{\zerodisplayskips}
\setlength\parindent{0pt}
\setlist[enumerate]{itemsep=2pt, topsep=0pt, parsep=3pt}

\begin{document}
\small
\begin{multicols*}{4}
    \boxed{\text{Done by Marcus Pang in 2021}}
    \customsection{Statements}
    \begin{enumerate}[wide, labelindent=2pt]
        \item Universal (all, every, any): $\forall x\in D (Q(x))$ is \true\ if and only if $Q(x)$ is \true\ for every $x\in D$
        \item Conditional (if...then): $\forall x\in D\ (P(x) \implies Q(x))$ is \true\ if and only if $({\sim} P(x) \Or Q(x))$ is \true\ for every $x\in D$
        \item Existential (there exists, there is, some): $\exists x\in D$ such that $Q(x)$ is \true\ for at least one $x\in D$
    \end{enumerate}

    \customsection{Definitions}
    \begin{enumerate}[wide, labelindent=2pt]
        \item Divisibility: if $n,d\in \Z \and n\neq 0$,
              \begin{align*}
                  d\ |\ n \iff \exists k\in\Z(n=dk)
              \end{align*}
        \item Congruence: if $a,b\in\Z \and n\in\Z^+$,
              \begin{align*}
                  a\equiv b\pmod{n} \iff n\mid (a-b)
              \end{align*}
        \item Rational:
              \begin{align*}
                  n\in\Q \iff \exists a,b\in\Z \left(n=\dfrac{a}{b} \and b \neq 0\right)
              \end{align*}
        \item Even:
              \begin{align*}
                  n \text{ is even } \iff \exists k \text{ such that }n=2k
              \end{align*}
        \item Odd:
              \begin{align*}
                  n \text{ is odd } \iff \exists k \text{ such that }n=2k+1
              \end{align*}
        \item Prime:
              \begin{multline*}
                  \forall r,s\in\Z^+ (n=rs\implies \\ (r=1\and s=n) \Or (r=n \and s=1))
              \end{multline*}
        \item Composite:
              \begin{multline*}
                  \exists r,s\in\Z^+ ((n=rs) \and (1<r<n)\\ \and (1<s<n))
              \end{multline*}
        \item Fraction in lowest term: largest integer that divides numerator and denominator is 1
        \item Negation of $p$ is ${\sim} p$
        \item Conjuction of $p$ and $q$ is $p \and q$
        \item Disjunction of $p$ and $q$ is $p \Or q$
        \item Statement/Propositional form: expression made of statement variables and logical connectives that becomes a statement when actual statements are substituted for the component statement variables
        \item Logical equivalence: statements with identical truth values for each possible substitution of statements for their statement variables
        \item Tautology/Contradiction: statement form that is always \true/\false\ regardless of the truth values of the individual statements substituted for its statement variables
        \item Conditional (sufficient condition, only if) of $q$ by $p$ is "if $p$ then $q$" or "$p$ implies $q$" denoted as
              \begin{align*}
                  p \implies q
              \end{align*}$p$ is the hypothesis/antecedent and $q$ is the conclusion/consequent.
        \item Contrapositive (sufficient condition, only if):
              \[{\sim} q \implies {\sim} p \equiv p \implies q\]
        \item Converse (necessary condition, if):
              \[q \implies p\]
        \item Inverse (necessary condition, if):
              \[{\sim} p \implies {\sim} q \equiv q \implies p\]
        \item Biconditional (if and only if):
              \[p \iff q\equiv (p\implies q) \and (q\implies p)\]
        \item Argument: sequence of statements, where all statements (except the last one) are called premises/assumptions/hypothesis. The final statement is called the conclusion.
              \begin{enumerate}
                  \item Syllogism: argument with 2 premises and 1 conclusion
                  \item Critical Row: row in truth table where all premises are \true
                  \item If there is a critical row in which the conclusion is \false, then the argument form is invalid
                  \item If for all critical rows the conclusion is \true, then the argument form is valid.
                  \item Sound if and only if it is valid and all its premises are true
              \end{enumerate}
        \item Predicate: sentence that contains a finite number of variables and becomes a statement when specific values are substituted for the variables, $P(x_1, x_2, \ldots)$
        \item Domain of a predicate variable: set of all values that may be substituted in place of the variable
        \item Truth set of $P(x)$: set of all elements of $D$ that make $P(x)$ \true\ when substituted for $x$, $\{x \in D\ |\ P(x)\}$
        \item Set: unordered collection of objects
              \begin{enumerate}
                  \item Sets of size 1 are called singletons
                  \item A set is finite if it has finitely many distinct elements
              \end{enumerate}
        \item Set Equality: sets with all the same elements
              \begin{align*}
                  A=B \iff \forall z(z\in A\iff z \in B)
              \end{align*}
        \item Subset: set where all elements are contained by another set (set included by another set)
              \begin{align*}
                  A\subseteq B \iff \forall z(z\in A\implies z\in B)
              \end{align*}
        \item Power set: the set of all subsets, $\P(A)$
        \item Cardinality: the number of (distinct) elements in a set, $|A|$
        \item Ordered pairs: expressions of the form $(x,y)$
        \item Ordered $n$-tuples: $(x_1, x_2, \ldots, x_n), n\geq 2$
              \begin{multline*}
                  (x_1,x_2,\ldots, x_n)=(y_1,y_2,\ldots, y_n) \iff \\ x_1=y_1 \and x_2=y_2 \and \ldots x_n=y_n
              \end{multline*}
        \item Cartesian product of sets:
              \begin{align*}
                  A\times B=\{x\in A \and x\in B\}
              \end{align*}
        \item Union of sets:
              \begin{align*}
                  A \cup B=\{x \mid x\in A\Or x\in B\}
              \end{align*}
        \item Intersection of sets:
              \begin{align*}
                  A \cap B=\{x\mid x\in A \and x\in B\}
              \end{align*}
        \item Complement of a set in another set:
              \begin{align*}
                  A-B=A\backslash B & =\{x\mid x \in A \and x\notin B\} \\
                  \overline{B}      & =U\backslash B
              \end{align*}
        \item Disjoint sets: $A\cup B=\varnothing$
        \item Pairwise/Mutually disjoint sets: for all distinct
              $i,j=\{1,2,\ldots,n\}$,
              \[
                  A_i\cap A_j=\varnothing
              \]
        \item Partition of set: $\C$ is a partition of a set $A$ if it is a set of mutually disjoint nonempty subsets (components) of $A$ whose union is $A$
              \begin{enumerate}
                  \item $\forall S\in\C(\varnothing\neq S\subseteq A)$
                  \item $\forall x \in A\ \exists S \in \C  (x\in S)\and$
                        \begin{multline*}
                            \forall x\in A\ \forall S_1,S_2\in\C \\ (x\in S_1\and x\in S_2 \implies S_1=S_2)
                        \end{multline*}
              \end{enumerate}
              \[
                  \therefore \forall x \in A\ \exists ! S\in\C (x\in S)
              \]
        \item Relation:
              \begin{enumerate}
                  \item $R\subseteq  A\times B$
                  \item $x\ R\ y \text{ for }(x,y)\in R \and x \centernot{R} y \text{ for }(x,y)\notin R$
              \end{enumerate}
              \begin{align*}
                  R      & = \{(x,y)\in A\times B \mid x\ R\ y\}      \\
                  R^{-1} & = \{(y,x)\in B\times A \mid y\ R^{-1}\ x\}
              \end{align*}
        \item Binary relation on a set: relation from $A$ to $A$
        \item Reflexive: $\forall x\in A(x\ R\ x)$
        \item Symmetric: $\forall x,y\in A(x\ R\ y\implies y\ R\ x)$
        \item Transitive: $\forall x,y,z\in A(x\ R\ y \and y\ R\ z \implies x\ R\ z)$
        \item Anti-symmetric:
              \[
                  \forall x,y\in A \left(x\ R\ y\and y\ R\ x\implies x=y\right)
              \]
        \item Comparable: $x\ R\ y \Or y\ R\ x$
        \item Equivalence relation: relation that is reflexive, symmetric, and transitive, usually ${\sim}$
        \item Equivalence class of $x$ w.r.t ${\sim}$: set of all elements that are ${\sim}$-related to $x$
              \[
                  [x]_{\sim}=\{y\in A\mid x{\sim} y\}
              \]
        \item Set of all equivalence classes (quotient of set by relation):
              \[
                  A/{\sim} =\{[x]_{\sim}\mid x\in A\}
              \]
        \item Representative of an equivalence class: element of the equivalence class
        \item Quotient $\Z/{\sim}_n$ is denoted as $\Z_n$ or $\Z/n\Z$, and addition/multiplication is defined as follows
              \begin{align*}
                  [x]+[y]=[x+y] &  & [x]\cdot[y]=[x\cdot y]
              \end{align*}
        \item (Non-strict) Partial Order: relation that is reflexive, anti-symmetric, and transitive, denoted with $\preccurlyeq$ and $x\prec y\equiv x\preccurlyeq y \and x\neq y$
        \item (Non-strict) Total/Linear Order: partial order where every pair of elements is comparable
        \item Partially Ordered Set (Poset) refers to the ordered pair $(A, R)$ where $R$ is a partial order on $A$
        \item Hasse Diagram of $\preccurlyeq$: if $x\prec y$ and no $z\in A$ such that $x\prec z\prec y$, then $x$ is placed below $y$ and a line joins $x$ to $y$, else no line
        \item Minimal Element: $\forall x\in A(x\preccurlyeq c\implies c = x)$
        \item Maximal Element: $\forall x\in A(c\preccurlyeq x\implies c = x)$
        \item Smallest Element: $\forall x\in A(c\preccurlyeq x)$
        \item Largest Element: $\forall x\in A(x\preccurlyeq c)$
        \item Linearization of $\preccurlyeq$ is a total order $\preccurlyeq^*$ such that
              \[\forall x,y\in A (x\preccurlyeq y \implies x \preccurlyeq^* y)\]
        \item Function or a map from $A$ to $B$: assignment to each element of $A$ exactly one element of $B$, $f:A\rightarrow B$
              \begin{enumerate}
                  \item If $x\in A$ then $f(x)$ is the image of $x$ under $f$, if $y=f(x)$ then $f$ maps $x$ to $y$, denoted as $f:x\mapsto y$
                  \item $A$ is the domain of $f$, $B$ is the codomain of $f$
              \end{enumerate}
        \item Identity function: id$:A\rightarrow A$, which satisfies
              \[
                  \forall x\in A \left(\text{id}_A(x)=x\right)
              \]
    \end{enumerate}

    \customsection{Order of Operations}
    \begin{enumerate}[wide, labelindent=2pt]
        \item ${\sim}$
        \item $\and$ and $\Or$
        \item $\implies$ and $\iff$
    \end{enumerate}

    \customsection{Argument Forms}
    \noindent\textbf{Rules of Inferences}
    \[
        \text{premise}_1\and \ldots\and \text{premise}_n \implies \text{conclusion}
    \]
    \begin{enumerate}[wide, labelindent=2pt]
        \item Modus Ponens and Universal Modus Ponens
              \vspace{-4pt} \par {\centering
                  $p \implies q$ \\
                  $p$           \\
                  $\therefore q $ \par
              }
        \item Modus Tollens and Universal Modus Tollens
              \vspace{-4pt} \par {\centering
              $p \implies q$ \\
              ${\sim} q$           \\
              $\therefore\ {\sim} p $ \par
              }
        \item Generalization
              \begin{align*}
                  p                   &  & q                  \\
                  \therefore p  \Or q &  & \therefore p \Or q
              \end{align*}
        \item Specialization
              \begin{align*}
                  p \and q     &  & p \and q     \\
                  \therefore p &  & \therefore q
              \end{align*}
        \item Conjunction
              \vspace{-4pt} \par {\centering
                  $p$ \\
                  $q$           \\
                  $\therefore p \and q $ \par
              }
        \item Elimination
              \begin{align*}
                  p \Or q      &  & p \Or q      \\
                  {\sim} q     &  & {\sim} p     \\
                  \therefore p &  & \therefore q
              \end{align*}
        \item Transitivity
              \vspace{-4pt} \par {\centering
                  $p\implies q$ \\
                  $q \implies r$ \\
                  $\therefore p \implies r $ \par
              }
        \item Proof by division into cases
              \vspace{-4pt} \par {\centering
                  $p\Or q$ \\
                  $p\implies r$ \\
                  $q\implies r$ \\
                  $\therefore r $ \par
              }
        \item Contradiction rule\\
              \vspace{-16pt} \par {\centering
                  ${\sim} p \implies \false$ \\
                  $\therefore p $ \par
              }
    \end{enumerate}
    \noindent\textbf{Rules of Inference for Quantified Statements}
    \begin{enumerate}[wide, labelindent=2pt]
        \item Universal Modus Ponens
              \par {\centering
                  $\forall x\in D (P(x)\implies Q(x))$\\
                  $P(a) \text{ for a particular a}\in D$\\
                  $\therefore Q(a)$\par
              }
        \item Universal Modus Tollens
              \par {\centering
                  $\forall x\in D (P(x)\implies Q(x))$        \\
                  ${\sim} Q(a) \text{ for a particular a}\in D$ \\
                  $\therefore {\sim} P(a)$ \par
              }
        \item Universal Transitivity
              \par {\centering
                  $\forall x\in D (P(x)\implies Q(x))$        \\
                  $\forall x\in D (Q(x)\implies R(x))$        \\
                  $\therefore \forall x (P(x)\implies R(x))$ \par
              }
        \item Universal Instantiation
              \par {\centering
                  $\forall x\in D(P(x))$        \\
                  $\therefore P(a)$ if $a\in D$ \par
              }
        \item Universal Generalization
              \par {\centering
                  $\forall P(a)$ for every $a\in D$        \\
                  $\therefore \forall x\in D (P(x))$ \par
              }
        \item Existential Instantiation
              \par {\centering
                  $\exists x\in D (P(x))$        \\
                  $\therefore P(a)$ for some $a\in D$ \par
              }
        \item Existential Generalization
              \par {\centering
                  $P(a)$ for some $a\in D$        \\
                  $\therefore \exists x\in D (P(x))$ \par
              }
    \end{enumerate}

    \noindent\textbf{Common Fallacies}
    \begin{enumerate}[wide, labelindent=2pt]
        \item Ambiguous premises
        \item Circular reasoning
        \item Jumping to a conclusion
        \item Converse/Inverse error
    \end{enumerate}

    \customsection{Proof Types}
    \begin{enumerate}[wide, labelindent=2pt]
        \item Direct Proof: using algebra/definitions to construct an argument
        \item By Construction: form of direct proof which comes up with a specific example to prove/disprove the statement
        \item By Contradiction: assume the negation of the statement and arrive to the conclusion that the negation is false, and since every step is logically correct, the assumption must be false
        \item By Exhaustion: list all possible scenarios (for finite cases)
    \end{enumerate}

    \customsection{(T2.1.1) Logical Equivalences}
    \begin{enumerate}[wide, labelindent=2pt]
        \item Commutative Law
              \begin{align*}
                  p \and q\equiv q \and p &  & p \Or q\equiv q \Or p
              \end{align*}
        \item Associative laws
              \begin{align*}
                  p \and q \and r \equiv (p \and q) \and r & \equiv p \and (q \and r) \\
                  p \Or q \Or r \equiv (p \Or q) \Or r     & \equiv p \Or (q \Or r)
              \end{align*}
        \item Distributive laws
              \begin{align*}
                  p \and (q \Or r) & \equiv (p \and q) \Or (q \and r) \\
                  p \Or (q \and r) & \equiv (p \Or q) \and (q \Or r)
              \end{align*}
        \item Identity laws
              \begin{align*}
                  p \and \true \equiv p &  & p \Or \false  \equiv p
              \end{align*}
        \item Negation laws
              \begin{align*}
                  p\ \Or {\sim} p\equiv \true &  & p\ \and {\sim} p  \equiv \false
              \end{align*}
        \item Double negative law
              \begin{align*}
                  {\sim}({\sim} p) & \equiv p
              \end{align*}
        \item Idempotent laws
              \begin{align*}
                  p \and p \equiv p &  & p \Or p \equiv p
              \end{align*}
        \item Universal bound laws
              \begin{align*}
                  p \Or \true \equiv \true &  & p \and \false \equiv \false
              \end{align*}
        \item De Morgan’s laws
              \begin{align*}
                  {\sim}(p \and q) & \equiv {\sim} p \Or {\sim} q  \\
                  {\sim}(p \Or q)  & \equiv {\sim} p \and {\sim} q
              \end{align*}
        \item Absorption laws
              \begin{align*}
                  p \Or (p \and q) \equiv p &  &
                  p \and (p \Or q) \equiv p
              \end{align*}
        \item Negation of true and false
              \begin{align*}
                  {\sim} \true \equiv \false &  & {\sim} \false \equiv \true
              \end{align*}

    \end{enumerate}

    \customsection{(T5.3.5) Set Identities}
    \begin{enumerate}[wide, labelindent=2pt]
        \item Identity Law
              \begin{align*}
                  A \cup \varnothing = A &  & A\cap U = A
              \end{align*}
        \item Universal Bound Law
              \begin{align*}
                  A \cup U = U &  & A \cap \varnothing = \varnothing
              \end{align*}
        \item Idempotent Law*
        \item Double Complement Law
              \begin{align*}
                  \overline{(\overline{A})}=A
              \end{align*}
        \item Commutative Law*
        \item Associative Law*
        \item Distributive Law*
        \item De Morgan's Law*
        \item Absorption Law*
        \item Complement Law
              \begin{align*}
                  A\cup \overline{A}=U &  & A\cap \overline{A}=\varnothing
              \end{align*}
        \item Set Difference Law
              \begin{align*}
                  A\backslash B=A\cap \overline{B}
              \end{align*}
        \item Top and Bottom Law
              \begin{align*}
                  \overline{\varnothing}=U &  & \overline{U}=\varnothing
              \end{align*}
    \end{enumerate}
    * --- see logical equivalence

    \customsection{Theorems ($5^{th}$ edition)}
    \begin{enumerate}[wide, labelindent=2pt]
        \item (T4.8.1) $\sqrt{2}$ is irrational
        \item (P4.7.4) $\forall n\in\Z (n^2$ is even $\implies n$ is even)
        \item (T3.2.1) ${\sim} (\forall x\in D, P(x)) \equiv \exists x\in D$ such that $P(x)$
        \item (T3.2.2) ${\sim} (\exists x\in D$ such that $P(x)) \equiv \forall x\in D, P(x)$
        \item (T4.3.1) Every integer is a rational number
        \item (T4.3.2) Sum of any two rational number is rational
        \item (C4.2.3) Double of a rational number is rational
        \item (T4.4.1) $\forall a,b\in\Z^+ (a\ |\ b\implies a\leq b)$
        \item (T4.4.2) The only divisors of 1 are 1 and -1
        \item (T4.4.3) $\forall a,b,c\in\Z (a\ |\ b \and b\ |\ c\implies a\ |\ c)$
        \item (T4.7.1) There is no greatest integer
        \item (T5.1.1.7) Empty set is a unique set with no element
        \item (T5.2.4) For finite sets, $|\P(A)|=2^{|A|}$
        \item (T5.3.12) For (pairwise) disjoint sets,
              \begin{align*}
                  |A\cup B|                         & =|A|+|B|                  \\
                  |A_1\cup A_2\cup \ldots \cup A_n| & =|A_1|+|A_2|+\ldots+|A_n|
              \end{align*}
        \item (T5.3.13) Inclusion-Exclusion Principle: for finite sets, $|A\cup B|=|A|+|B|-|A\cap B|$
        \item (L6.4.4) Let ${\sim}$ be an equivalence relation, then
              \[
                  \begin{tikzcd}
                      & x \sim y \arrow[dr, Rightarrow] &\\
                      {[x]_{\sim}}={[y]_{\sim}} \arrow[ur, Rightarrow] & & \arrow[ll, Rightarrow] [x]\cap[y]\neq\varnothing
                  \end{tikzcd}
              \]
        \item (T6.4.9) $A/{\sim}$ is a partition of $A$
        \item (P7.1.5) Addition/Multiplication is well-defined on $\Z_n$. $\forall n\in\Z^+$,
              \begin{multline*}
                  [x_1]=[x_2] \and [y_1]=[y_2] \implies \\
                  [x_1]+[y_1]=[x_1+y_1]=[x_2+y_2]=[x_2]+[y_2] \\
                  \and [x_1]\cdot [y_1]=[x_1\cdot y_1]=[x_2\cdot y_2]=[x_2]\cdot [y_2]
              \end{multline*}
        \item (P7.4.4) For posets, a smallest element is minimal and there is at most one smallest element
        \item (P7.4.6) For a nonempty finite poset, a minimal element can be found
        \item (T7.4.10) For any partial order on a set, there exists a total order on that set
        \item (Tutorial 4) Division Theorem:
              \begin{multline*}
                  \forall n\in\Z, d\in\Z^+ \exists !q,r\in\Z \text{ s.t. } \\
                  n=dq+r \and 0\leq r < d
              \end{multline*}
        \item (P6.2.16) ${\sim}_\C$ the same-component relation is an equivalence relation
        \item (P6.3.4) ${\sim}_n$ the congruence-mod-$n$ relation is an equivalence relation

    \end{enumerate}

    \customsection{Set Notations}
    \begin{enumerate}[wide, labelindent=2pt]
        \item Roster notation: $\{x_1, x_2, ..., x_n\}$
        \item Set builder notation: $\{x\in U\ |\ P(x)\}$
        \item Replacement notation: $\{f(x)\ |\ x\in U\}$
    \end{enumerate}

    \customsection{Useful}
    \begin{enumerate}[wide, labelindent=2pt]
        \item $(p \Or q) \and {\sim}(p \and q)\equiv p \oplus q$
        \item $(A\cap B)\cup (A\backslash B)=A$
        \item $A\cap B\subseteq A$
        \item (Tutorial 4) The following are equivalent:
              \begin{enumerate}
                  \item $\forall x, y\in A (x\ R\ y\implies y\ R\ x)$
                  \item $R=R^{-1}$
                  \item $\forall x,y\in A(x\ R\ y\iff y\ R\ x)$
              \end{enumerate}
        \item (Tutorial 4) Relation on $\Q:$
              \begin{enumerate}
                  \item $xy\geq 0$ is reflexive and symmetric
                  \item $xy\ge 0$ is symmetric and transitive
              \end{enumerate}
        \item (Quiz 3)
              \begin{enumerate}
                  \item Order of quantifiers can be freely arranged if all quantifiers are of the same type
                  \item $\forall (P(x) \and Q(x)) \iff \forall x P(x)\and \forall x Q(x)$
                  \item $\exists (P(x) \Or Q(x)) \iff \exists x P(x)\Or \exists x Q(x)$
              \end{enumerate}
        \item (Quiz 5)
              \begin{enumerate}
                  \item $\P(\varnothing)=\{\varnothing\}$ has 1 element and 2 subsets
                  \item For all sets, $B\times A \neq A \times B$
              \end{enumerate}
        \item (Quiz 6)
              \begin{enumerate}
                  \item non-symmetric relations which are reflexive $\centernot\implies$ they are antisymmetric
                  \item non-symmetric relation which are antisymmetric $\centernot\implies$ they are reflexive
                  \item symmetric relations may be antisymmetric or not
                  \item antisymmetric relations may be symmetric or not
                  \item $\min(|R|)=n$ if $R$ is an equivalence relation on $A$ where $|A|=n$
              \end{enumerate}
        \item (Quiz 7.3/7.4)
              \begin{enumerate}
                  \item not symmetric may not be antisymmetric
                  \item for finite posets,
                        \begin{enumerate}[wide]
                            \item any minimal element is not necessarily smallest
                            \item any smallest is minimal
                            \item the smallest element is unique
                            \item $\exists$ smallest $implies$ exactly 1 minimal
                            \item exactly 1 minimal $\centernot\implies$ $\exists$ smallest
                        \end{enumerate}
              \end{enumerate}
    \end{enumerate}

    \customsection{Assumptions}
    \begin{enumerate}[wide, labelindent=2pt]
        \item every integer is even or odd, but not both
        \item every rational can be reduced to a fraction in its lowest term
    \end{enumerate}

\end{multicols*}
\end{document}