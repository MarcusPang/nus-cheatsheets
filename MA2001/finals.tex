\documentclass[landscape, a4paper]{article}
\usepackage[utf8]{inputenc}
\usepackage[T1]{fontenc}
\usepackage{multicol}
\usepackage{wrapfig}
\usepackage[top=2mm,bottom=2mm,left=2mm,right=2mm]{geometry}
\usepackage[framemethod=tikz]{mdframed}
\usepackage{pdfpages}
\usepackage{amsmath}
\usepackage{amsthm}
\usepackage{amssymb}
\usepackage{amsfonts}
\usepackage{mathrsfs}
\usepackage{tikz-cd}
\usepackage{enumitem}
\usepackage{centernot}
\usepackage{ifthen}

% \DeclarePairedDelimiter\ceil{\lceil}{\rceil}
% \DeclarePairedDelimiter\floor{\lfloor}{\rfloor}
\let\leq\leqslant
\let\geq\geqslant
\newcommand{\Z}{\mathbb{Z}}
\newcommand{\Q}{\mathbb{Q}}
\newcommand{\R}{\mathbb{R}}
\newcommand{\V}[1]{\textbf{\textit{#1}}}
\newcommand{\C}{\mathscr{C}}
\newcommand{\rank}{\operatorname{rank}}
\newcommand{\adj}[1]{\operatorname{adj}\left(#1\right)}
\newcommand{\nullity}{\operatorname{nullity}}
\newcommand{\norm}[1]{\left\lVert#1\right\rVert}
\DeclareMathSymbol{\perp}{\mathrel}{symbols}{"3F}
\newcommand{\Or}{\vee}
\newcommand{\true}{\textbf{true}}
\newcommand{\false}{\textbf{false}}
\renewcommand{\P}{\mathcal{P}}
\renewcommand{\and}{\wedge}
\newcommand{\Span}{\operatorname{span}}
\newcommand{\zerodisplayskips}{%
  \setlength{\abovedisplayskip}{0pt}%
  \setlength{\belowdisplayskip}{0pt}%
  \setlength{\abovedisplayshortskip}{0pt}%
  \setlength{\belowdisplayshortskip}{0pt}
}
\newcommand{\customsection}[1]{
    \vspace*{-10pt}
    \section*{#1}
    % \vspace*{-2pt}
}
\newcommand{\heading}[1]{
    \noindent\textbf{#1}
}

\appto{\normalsize}{\zerodisplayskips}
\appto{\small}{\zerodisplayskips}
\appto{\footnotesize}{\zerodisplayskips}
\setlength\parindent{0pt}
\setlist[enumerate]{itemsep=2pt, topsep=2pt, parsep=3pt}
\setlist[description]{itemsep=2pt, topsep=0pt, parsep=3pt}

\begin{document}
\small
\begin{multicols*}{4}
    \boxed{\text{Done by Marcus Pang in 2021}}
    \customsection{Things to Note}
    \heading{Matrix Properties}
    \begin{enumerate}
        \item $\V{AB}\neq \V{BA}$
        \item $\V{AB}=\textbf{0}\centernot\implies \V{A}=\textbf{0}\Or \V{B}=\textbf{0}$
        \item $(\V{AB})^n\neq \V{A}^n\V{B}^n$
        \item $\V{A}\adj{\V{A}}=\det{\V{A}}\cdot\V{I}$
        \item REF has zero row $\implies$ not invertible
    \end{enumerate}
    \heading{Cramer's rule}
    $$
        \V{x}=\dfrac{1}{\det\V{A}}\begin{pmatrix}
            \det\V{A}_1 \\
            \det\V{A}_2 \\
            \vdots      \\
            \det\V{A}_n
        \end{pmatrix}
    $$
    where $\V{A}_i=\V{A}$ with the $i$-th column replaced by \V{b}
    \heading{Spans, Subspaces, Linear Independence}
    \begin{enumerate}
        \item $\forall S\subseteq V, \dim V=n$
              \begin{enumerate}
                  \item $|S|=k<n\implies \Span S \neq V$
                  \item $|S|=k>n\implies S$ is linearly dependent
              \end{enumerate}
        \item check subset $V$ is subspace
              \begin{enumerate}
                  \item must contain zero vector
                  \item $\forall \V{u},\V{v}\in V\implies c\V{u}+d\V{v}\in V$
              \end{enumerate}
        \item $\textbf{0}\in S$ then $S$ is linearly dependent
        \item $\{\textbf{0}\}$ is a subspace of $\R^3$, with basis $\varnothing$
    \end{enumerate}
    \heading{Ranks}
    \begin{enumerate}
        \item $\rank \V{A} = \rank (\V{A} \mid \V{b}) \implies (\V{A} \mid \V{b})$ is consistent
        \item $\rank \V{AB} \leq \min\{\rank \V{A}, \rank \V{B}\}$
        \item $\V{B}$ is invertible $\implies \rank \V{BA}=\rank \V{A}$ (converse is false)
    \end{enumerate}
    \heading{Orthogonal}
    \begin{enumerate}
        \item orthogonal set $\implies$ linearly independent set
        \item \V{p} is the projection of \V{u} onto $V$ if $\V{u} - \V{p}$ is orthogonal to $V$
        \item orthogonal matrix means the transpose is its inverse
        \item rows/cols of orthogonal matrix is a basis for $\R^n$
    \end{enumerate}
    \heading{Diagonalization}
    \begin{enumerate}
        \item matrix is diagonlizable $\iff$ it has $n$ linearly independent eigenvectors $\iff \dim(E_{\lambda_i})=r_i$
    \end{enumerate}
    \heading{Recurrence Relation}
    \begin{enumerate}
        \item recurrence matrix = $\begin{pmatrix}a_n\\a_{n+1}\end{pmatrix}=\begin{pmatrix}0&1\\p & q\end{pmatrix}$
        \item orthogonally diagonalizable $\iff$ symmetric
    \end{enumerate}
    \customsection{Questions}
    \heading{Chapter 1}
    \begin{enumerate}
        \item relative positions of three lines if there are
              \begin{enumerate}
                  \item no soln: either all parallel and not all are the same or two intersect at a point not on the third line
                  \item one soln: all different and intersect at a point or two same lines intersecting with third line at a point
                  \item infinite solns: all lines are the same
              \end{enumerate}
        \item relative positions of three planes if there are
              \begin{enumerate}
                  \item no soln: either all parallel and not all are the same or two intersect at a line parallel and not on the third plane
                  \item one soln: three planes intersect at a single point
                  \item infinite solns: two same planes intersect at a line on third plane or all three planes are the same
              \end{enumerate}
        \item inconsistent linear system can have more unknowns than equations
        \item a linear system with unique solution: no. of equations $\geq$ no. of unknowns
        \item linear system with infinite solutions can have more equations than unknowns
        \item matrix with last column as pivot column can be considered as RREF
        \item non-homogeneous system cannot have trivial solution
    \end{enumerate}
    \heading{Chapter 2}
    \begin{enumerate}
        \item $\V{Ax}=\textbf{0}$ has non-trivial soln $\implies \V{Ax}=\V{b}$ has either no soln or infinitely many soln
        \item \V{A} and \V{B} are invertible $\centernot\implies$ \V{A} + \V{B} is invertible
        \item \V{A} and \V{B} are singular $\centernot\implies$ \V{A} + \V{B} is singular
        \item if \V{A}, \V{B} are singular square matrices, \V{AB}, \V{BA} are singular
        \item for $\V{A}_{m\times n}, \V{B}_{n\times m}$, $m>n\implies \V{AB}$ is singular
    \end{enumerate}
    \heading{Chapter 3}
    \begin{description}
        \item[Q20] $\Span (S_1 \cap S_2)\neq \Span S_1 \cap \Span S_2$
        \item[Q20/23] $\Span (S_1 \cup S_2) = \Span S_1 + \Span S_2 \neq \Span S_1 \cup \Span S_2$
        \item[Q24] let $V, W$ be subspaces of $\R^n$
            \begin{enumerate}
                \item $V\cap W$ is a subspace of $\R^n$
                \item $V\cup W$ is a subspace of $\R^n\iff V\subseteq W\Or W\subseteq V$
            \end{enumerate}
        \item[Q26] for a nonzero matrix in REF, the nonzero rows are always linearly independent
        \item[Q30] $\V{Pu}_1, \V{Pu}_2, \hdots, \V{Pu}_k$ linearly independent \\
            $\implies \V{u}_1, \V{u}_2, \hdots, \V{u}_k$ linearly independent.
        \item[Q30] $\V{u}_1, \V{u}_2, \hdots, \V{u}_k$ linearly independent
            \begin{description}
                \item $P$ invertible $\implies \V{Pu}_1, \V{Pu}_2, \hdots, \V{Pu}_k$ linearly independent
                \item $P$ is singular $\centernot\implies \V{Pu}_1, \V{Pu}_2, \hdots, \V{Pu}_k$ linearly independent/dependent
            \end{description}
        \item[Q41] $S$ is a finite subset of $V$ such that
            \begin{description}
                \item $\Span S=V \implies \exists S'\subseteq S$ such that $S'$ is a basis for $V$
                \item $S$ is linearly independent $\implies \exists S'$ a basis for $V$ such that $S'\supseteq S$
            \end{description}
        \item[Q42] $\dim V=n\implies \exists n+1$ vectors such that $\forall \V{v}\in V$, \V{v} can be expressed as a linear combination of $n+1$ vectors
        \item[Q43] $\dim(V+W)=\dim(V) + \dim(W)- \dim(V\cap W)$
    \end{description}
    \heading{Chapter 4}
    \begin{description}
        \item[Q17] In $\R^3$,
            \begin{enumerate}
                \item $\rank = 0\implies$ soln set is $\R^3$
                \item $\rank = 1\implies$ soln set is plane in $\R^3$ passing through origin
                \item $\rank = 2\implies$ soln set is line in $\R^3$ passing through origin
                \item $\rank = 3\implies$ soln set is \{\textbf{0}\}
            \end{enumerate}
        \item[Q20] $\V{AB} = 0\implies \operatorname{col}(B)\subseteq \operatorname{null}(A)$
        \item[Q21] row space and nullspace of a matrix cannot contain the same non-zero vector
        \item[Q23] $\rank (\V{A} + \V{B}) \leq \rank\V{A} + \rank\V{B}$
        \item[Q25] $\rank \V{A}=\rank \V{A}^T\V{A}=\rank \V{AA}^T$
        \item $\nullity \V{A}=\nullity \V{A}^T\V{A}\neq \nullity \V{AA}^T$
    \end{description}
    \heading{Chapter 5}
    \begin{description}
        \item[Cauchy-Schwarz] $\norm{\V{u}\cdot \V{v}}\leq \norm{\V{u}}\norm{\V{v}}$
        \item[Triangle] $\norm{\V{u} + \V{v}}\leq \norm{\V{u}} + \norm{\V{v}}$
        \item $d(\V{u}, \V{w})\leq d(\V{u}, \V{v}) + d(\V{v}, \V{w})$
        \item[Q31] $S, T$ orthonormal bases for $V$\\
            $\implies \forall \V{u}, \V{v}\in V, (\V{u})_S \cdot (\V{v})_S = (\V{u})_T\cdot (\V{v})_T$
        \item[Q32] $\V{A}$ is orthogonal $\implies$
            \begin{description}
                \item[-] $\norm{\V{u}} = \norm{\V{Au}}$
                \item[-] $d(\V{u}, \V{v}) = d(\V{Au}, \V{Av})$
                \item[-] angle between $\V{u},\V{v}=$ angle between $\V{Au}, \V{Av}$
                \item[Q33] $S=\{\V{u}_1, \V{u}_2, \hdots, \V{u}_n\}$ is a basis for $\R^n\implies T=\{\V{Au}_1, \V{Au}_2, \hdots, \V{Au}_n\}$ is a basis for $\R^n$
                \item[Q33] $S$ orthogonal/orthonormal $\implies T$ orthogonal/orthonormal
            \end{description}
    \end{description}
    \heading{Chapter 6}
    \begin{description}
        \item[Q3] \V{A} is a square matrix, $\V{Ax}=\lambda\V{x}\implies \V{A}^T\V{u}=\lambda\V{u}$
        \item[Q23] $\V{A}$ diagonalizable $\implies \V{A}^T$ diagonalizable
        \item[Q23] $\V{A}, \V{B}$ diagonalizable
            \begin{description}
                \item $\centernot\implies \V{A}+\V{B}$ diagonalizable
                \item $\centernot\implies \V{AB}$ diagonalizable
            \end{description}
        \item[Q26] for symmetric matrices, eigenvectors of different eigenspaces are orthogonal
        \item[Q30] $\V{A}, \V{B}$ orthogonally diagonalizable
            \begin{description}
                \item $\implies \V{A}+\V{B}$ orthogonally diagonalizable
                \item $\centernot\implies \V{AB}$ orthogonally diagonalizable
            \end{description}
    \end{description}
    \heading{Assignments}
    \begin{description}
        \item[A1 Q7] for $\V{A}_{m\times n},\ \V{B}_{n\times m}$
            \begin{enumerate}[wide]
                \item $\V{A}$ singular $\implies \V{BA}$ singular
                \item $m>n\implies \V{AB}$ singular
                \item $\V{A}$ invertible $\centernot\implies \V{AB}, \V{BA}$ invertible
            \end{enumerate}
        \item[A3 Q3] for $\V{A}_{m\times n}, \V{B}_{n\times k}$
            \begin{enumerate}[wide]
                \item $m>n \and \rank\V{A}=n\implies \rank\V{AB}=\rank\V{B}$
                \item $k>n\and \rank\V{B}=n\implies \rank\V{AB}=\rank\V{A}$
            \end{enumerate}
        \item[A3 Q7] for some subspace $V$ of $\R^n$,
            \begin{enumerate}[wide]
                \item $V\cap V^\perp = \{\textbf{0}\}$
                \item $\V{A}_{n\times k}$ where $\text{col}(\V{A})=V \\ \implies  \text{null}(\V{AA}^T)=V^\perp$
            \end{enumerate}
        \item[A4 Q2] eigenvalues of orthogonal matrix are $\pm 1$
        \item[A4 Q2] $\V{A}$ diagonalizable orthogonal matrix (not orthogonally diagonalizable) then $\V{A}^2=\V{I}$
    \end{description}
    \customsection{Exams}
    \begin{description}
        \item[14/15 S1 Q6] $\forall \V{u},\V{v}\neq \textbf{0}$, $\Span\{\V{u}\}\cap \Span\{\V{v}\}=\Span\{\V{u},\V{v}\}\implies \Span\{\V{u}\}=\Span\{\V{v}\}$
        \item[14/15 S2 Q3] for $\V{A}_{m\times n}, \V{c}_{m\times 1}$, $n>m\implies \V{Ax}=\V{c}$ always has infinitely many least squares solutions
        \item[14/15 S2 Q4] for $\V{M}_{n\times n}, \V{N}_{n\times n}$ with the same $n$ linearly independent eigenvectors, $\V{MN}=\V{NM}$
        \item[15/16 S2 Q5] $T:\R^n \rightarrow \R^m$ such that standard matrix is diagonalizable $\implies \text{R}(T)=\text{R}(T\circ T)\and \ker(T)=\ker(T\circ T)$
        \item[16/17 S2 Q4] $T:\R^n\rightarrow \R^n$ such that $\ker(T)=\ker(T\circ T)\implies \ker(T\circ T) = \ker(T\circ T\circ T)$
        \item[16/17 S2 Q6] $\V{A}_{n\times n}, \V{B}_{n\times n}
                \\ \implies \nullity\V{A}+\nullity\V{B}\geq \nullity\V{AB}$
        \item[19/20 S1 Q6] $\V{A}_{n\times n}$ such that $\V{A}^2=\V{I}\implies \rank(\V{I} + \V{A})+\rank(\V{I}-\V{A})=n$
        \item[19/20 S1 Q6] $\V{A}^2-\V{B}^2=\V{AB}\implies \V{A}, \V{B}$ cannot be orthogonal
        \item[20/21 S1 Q2] $\forall \V{v}\in\R^n$, $\V{v}$ can be written \textbf{uniquely} as $\V{v}=\V{v}_1 + \V{v}_2$ where $\V{v}_1\in V, \V{v}_2\in V^\perp$
        \item[20/21 S1 Q4] $E_0$ corresponds to nullspace of matrix
            \item[20/21 S1 Q5]$T:\R^n\rightarrow\R^n$
            \begin{enumerate}
                \item[$\implies$] $R(T^{k+1})\subseteq R(T^k)$
                \item[$\implies$] $m>n\and T^m$ is zero transformation $\implies T^n$ is zero transformation
            \end{enumerate}
    \end{description}
\end{multicols*}
\end{document}